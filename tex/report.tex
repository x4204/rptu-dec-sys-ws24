\documentclass[a4paper]{article}

\usepackage[english]{babel}
\usepackage[utf8]{inputenc}
\usepackage{caption}
\usepackage{fancyhdr}
\usepackage{float}
\usepackage{listings}
\usepackage{pgfplots}
\usepackage{vmargin}
\usepackage{wrapfig}

% ------------------------------------------------------------------------------
% general
% ------------------------------------------------------------------------------
\setpapersize{A4}
\setmargins
  {2.5cm}{2.0cm}  % left and top margin
  {16cm}{22cm}    % text width and height
  {48pt}{36pt}    % header height and spacing
  {0pt}{30pt}     % footer height and spacing

\pagestyle{fancy}

\sloppy
\frenchspacing

\setlength{\parindent}{0pt}
\setlength{\parskip}{5pt}
\setlength{\fboxsep}{1.5mm}

% ------------------------------------------------------------------------------
% colors
% ------------------------------------------------------------------------------
% \definecolor{codegreen}{rgb}{0,0.6,0}
% \definecolor{codegray}{rgb}{0.5,0.5,0.5}
% \definecolor{codepurple}{rgb}{0.38,0,0.72}

% ------------------------------------------------------------------------------
% commands
% ------------------------------------------------------------------------------
\newcommand{\ceil}[1]{\left\lceil{} {#1} \right\rceil{}}
\newcommand{\floor}[1]{\left\lfloor{} {#1} \right\rfloor{}}

% ------------------------------------------------------------------------------
% `pgfplots` setup
% ------------------------------------------------------------------------------
\pgfplotsset{compat=1.18}

\usepgfplotslibrary{
  dateplot,
  statistics,
}

% ------------------------------------------------------------------------------
% `listings` setup
% ------------------------------------------------------------------------------
\lstset{
  numbers=none,
  basicstyle=\ttfamily,
  upquote=true,
}


\begin{document}
\lhead{
  \begin{tabular}{l}
    {\bf Decentralized Systems: IPFS Lab Report}\\
    {\bf Submitted by:} Maxim Rastorguev, Roman Popa
  \end{tabular}
}
\rhead{}


\section{Basic ideas of IPFS}

% https://docs.ipfs.tech/concepts/file-systems/#unix-file-system-unixfs
% ipfs add --help
% ipfs cat --help
\subsection{File upload, download and storage}

A file can be uploaded to IPFS by issuing \verb|ipfs add PATH...| on the
command line. The \verb|ipfs add| command is handled in
\verb|kubo/core/commands/add.go|, which later delegates everything to
\verb|UnixfsAPI.Add()|. There the file is split into smaller parts called
chunks, if necessary, and a CID is computed for every chunk. If the file is
small enough, only one chunk will be created. Additionally, layouting is
performed: file chunks are arranged into a Merkle DAG and uploaded. Metadata
about the chunks is stored in a UnixFS node. There are two strategies for
building the Merkle DAG: \verb|balanced| (default) and \verb|trickle|. When the
upload for all chunks is done, the CID of the root of the DAG is returned to
the user as a reference to the file.

A file can be downloaded from IPFS by issuing \verb|ipfs cat IPFS-PATH...| on
the command line. The CID is used to lookup the content in the DHT and find
IPFS instances that store the content linked by the CID. When downloading a
file, the Merkle DAG is traversed in in-order and the content of every chunk is
sent back. It is also possible to read only a particular byte range from the
file by specifying the offset and length.


\subsection{Peer connections}

Kubo has a hardcoded list of bootstrap nodes
(\verb|go-libp2p-kad-dht/dht_bootstrap.go| and
\verb|config/bootstrap_peers.go|). When launched, it will try to automatically
connect to them and fill its routing table with peers. From time to time (10
minutes by default), it will evict peers if not reachable
(\verb|go-libp2p-kad-dht/rtrefresh/rt_refresh_manager.go|). If an IPFS instance
is encountered while querying (\verb|go-libp2p-kad-dht/query.go| -
\verb|IpfsDHT.runQuery()|), Kubo will check if it is eligible to be added to
the routing table and add it (\verb|go-libp2p-kad-dht/dht.go| -
\verb|IpfsDHT.rtPeerLoop()|).


\section{Implementation}

...


\section{Results}

...


\end{document}
